% Generated slides for Genotype Coding
% Include this file in your main Beamer presentation

\begin{frame}{Section}
\centering
\Huge{Likelihood Ratio}
\end{frame}


\begin{frame}{Intuitional Description}
The likelihood ratio quantifies \textbf{how much more probable our observed data} is under one model compared to another, providing a natural way to choose between competing explanations of the data.
\end{frame}

\begin{frame}{Graphical Summary}
\includesvg[width=0.8\textwidth]{./cartoons/likelihood_ratio.svg}
\end{frame}


\begin{frame}{Key Formula}
The likelihood ratio between Model 1 and Model 2 is:

$$\text{LR} = \frac{L(\text{D}|\text{M}_1)}{L(\text{D}|\text{M}_2)}$$

Where:
\begin{itemize}
\item $\text{LR}$ is the likelihood ratio
\item $L(\text{D}|\text{M}_1)$ is the likelihood under Model 1
\item $L(\text{D}|\text{M}_2)$ is the likelihood under Model 2
\item $\text{D}$ represents the observed data
\end{itemize}
\end{frame}


\begin{frame}{Technical Details: Interpretation}
For the likelihood ratio (LR):
\begin{itemize}
\item $\text{LR} > 1$: Model 1 better explains the data (higher likelihood)
\item $\text{LR} < 1$: Model 2 better explains the data (higher likelihood)
\item $\text{LR} = 1$: Both models explain the data equally well
\end{itemize}

\end{frame}

\begin{frame}{Technical Details: Parameter $\theta$ in a model}

Suppose that, now instead of having two fully-specified models, we consider a model $M_1$ with a parameter space $\Theta$. 

The corresponding hypothesis ($H_1$) is often stated by saying that the parameter $\theta$ lies in a specified subset $\Theta_1$ of $\Theta$. And another hypothesis ($H_2$) which believes that $\theta$ lies in the $\Theta_2$. The likelihood ratio test statistic between those two hypothesis is given by:

$$
LR = \frac {~\sup_{\theta \in \Theta_1}{\mathcal{L}}(\theta)~}{~\sup_{\theta \in \Theta_2}{\mathcal{L}}(\theta)~}
$$

Here, the $\sup$ notation refers to the supremum. If $\Theta_1$ is a subset of $\Theta_2$, e.g., $\Theta_2 = \Theta$, as the constrained maximum cannot exceed the unconstrained maximum, the likelihood ratio is bounded between zero and one. 

\end{frame}

\begin{frame}{Technical Details: Properties}
\begin{itemize}
\item The likelihood ratio is always non-negative: $\text{LR} \geq 0$
\item Can be used to compare any two models/hypothesis
\item Forms the basis for many statistical tests and model selection criteria
\end{itemize}
\end{frame}

