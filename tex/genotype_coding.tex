% Generated slides for Genotype Coding
% Include this file in your main Beamer presentation

\begin{frame}{Section}
\centering
\Huge{Genotype Coding}
\end{frame}


\begin{frame}{Intuitional Description}
Genotype coding converts DNA nucleotide pairs into numerical values for statistical analysis, with the additive model being particularly valuable because genetic effects often accumulate proportionally with each additional copy of a variant.

\end{frame}

\begin{frame}{Graphical Summary}
\includesvg[width=0.8\textwidth]{./cartoons/genotype_coding.svg}
\end{frame}


\begin{frame}{Key Formula}
We use $\mathbf{X}_\text{raw}$ to denote the \textbf{Raw Genotype Matrix} (an $N \times M$ matrix representing genotypes for $N$ individuals at $M$ variants)
   
For diploid organisms, using different coding models:
\begin{itemize}
\item Additive model: takes value from $\{0, 1, 2\}$ (count of alternative alleles)
\item Dominant model: takes value from $\{0, 1\}$ (1: presence of alternative allele)
\item Recessive model: takes value from $\{0, 1\}$ (two copies of alternative allele required for 1)
\end{itemize}

\textbf{Standardized Genotype Matrix} $\mathbf{X}$: Normalized version of $\mathbf{X}_\text{raw}$

For each column $j$ (variant):
   $$X_{ij} = \frac{X_{\text{raw},ij} - \mu_j}{\sigma_j}$$
   
   Where:
\begin{itemize}
\item $\mu_j = \frac{1}{N}\sum_{i=1}^{N} X_{\text{raw},ij}$ (mean of variant $j$)
\item $\sigma_j = \sqrt{\frac{1}{N}\sum_{i=1}^{N} (X_{\text{raw},ij} - \mu_j)^2}$ (standard deviation of variant $j$)
\end{itemize}
\end{frame}


\begin{frame}{Technical Details: Three models}

For diploid organisms (with two copies of each chromosome, e.g., human), we use different coding models:

\begin{itemize}
\item \textbf{Additive model}: takes value from $\{0, 1, 2\}$
\begin{itemize}
\item 0: Homozygous for reference allele (AA)
\item 1: Heterozygous (Aa)
\item 2: Homozygous for alternative allele (aa)
\item Represents the count of alternative alleles
\end{itemize}
\end{itemize}
   
\begin{itemize}
\item \textbf{Dominant model}: takes value from $\{0, 1\}$
\begin{itemize}
\item 0: Homozygous for reference allele (AA)
\item 1: Either heterozygous or homozygous for alternative allele (Aa or aa)
\item Represents the presence of at least one copy of the alternative allele
\end{itemize}
\end{itemize}
   
\begin{itemize}
\item \textbf{Recessive model}: takes value from $\{0, 1\}$
\begin{itemize}
\item 0: Either homozygous for reference allele or heterozygous (AA or Aa)
\item 1: Homozygous for alternative allele (aa)
\item Represents when both copies of the alternative allele are present
\end{itemize}
\end{itemize}

\end{frame}

\begin{frame}{Technical Details: Standardized Genotype Matrix}

$\mathbf{X}$ is a normalized version of $\mathbf{X}_\text{raw}$ where each variant is scaled to have mean 0 and variance 1.

For each column $j$ (variant):
\begin{itemize}
\item $\mu_j = \frac{1}{N}\sum_{i=1}^{N} X_{\text{raw},ij}$ (mean of variant $j$)
\begin{itemize}
\item Represents the average number of alternative alleles at variant $j$ in the sample
\end{itemize}
\end{itemize}

\begin{itemize}
\item $\sigma_j = \sqrt{\frac{1}{N}\sum_{i=1}^{N} (X_{\text{raw},ij} - \mu_j)^2}$ (standard deviation of variant $j$)
\begin{itemize}
\item Measures the variability in the number of alternative alleles at variant $j$
\end{itemize}
\end{itemize}

This standardization ensures that:
\begin{itemize}
\item Each column of $\mathbf{X}$ has mean 0
\item Each column of $\mathbf{X}$ has variance 1
\item The standardized values reflect the deviation from the population mean in units of standard deviation
\end{itemize}
\end{frame}

