% Generated slides for Genotype Coding
% Include this file in your main Beamer presentation

\begin{frame}{Section}
\centering
\Huge{Minor Allele Frequency}
\end{frame}


\begin{frame}{Intuitional Description}
The minor allele frequency (MAF) represents the proportion of the less common allele in a population, which equals half the expected genotype value in diploid organisms like humans since each individual carries two alleles per locus.
\end{frame}

\begin{frame}{Graphical Summary}
\includesvg[width=0.8\textwidth]{./cartoons/minor_allele_frequency.svg}
\end{frame}


\begin{frame}{Key Formula}
$$\text{MAF}_j = \frac{\mathbb{E}[X_{\text{additive},j}]}{2} = \frac{1}{2N}\sum_{i=1}^{N} X_{\text{additive},ij}$$

Where:
\begin{itemize}
\item $X_{\text{additive},ij}$ represents the count of alternative alleles (0,1,2) for individual $i$ at variant $j$
\item The division by 2 is necessary because in the additive model for diploid organisms, each individual contributes two alleles
\end{itemize}
\end{frame}


\begin{frame}{Technical Details}
If there are only two alleles at the same locus, then the frequency of them can be denoted as $f_j$ and $1-f_j$, and the $\text{MAF}_j$ is always defined as $\min(f_j, 1 - f_j)$ (ensuring that it always represents the frequency of the \textbf{less} common allele in the population, i.e., \textbf{minor allele}). If there are more alleles, the \textbf{MAF} is specific for each minor allele.

\end{frame}

