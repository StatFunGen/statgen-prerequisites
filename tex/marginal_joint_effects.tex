% Generated slides for Genotype Coding
% Include this file in your main Beamer presentation

\begin{frame}{Section}
\centering
\Huge{Marginal Joint Effects}
\end{frame}


\begin{frame}{Intuitional Description}
Marginal effects measure a genetic variant's influence on a trait when considered \textbf{alone, ignoring other variants}, while joint effects measure each variant's independent contribution when all variants are \textbf{simultaneously included} in the model, revealing their true effects after \textbf{accounting for correlations (LD) between them}.
\end{frame}

\begin{frame}{Graphical Summary}
\includesvg[width=0.8\textwidth]{./cartoons/marginal_joint_effects.svg}
\end{frame}


\begin{frame}{Key Formula}
In multiple marker linear regression, we extend the single marker model to incorporate multiple genetic variants:

$$
\mathbf{Y} = \mathbf{X} \boldsymbol{\beta} + \boldsymbol{\epsilon}
$$

Where:
\begin{itemize}
\item $\mathbf{Y}$ is the $N \times 1$ vector of trait values for $N$ individuals
\item $\mathbf{X}$ is the $N \times M$ matrix of genotypes for $M$ variants across $N$ individuals
\item $\boldsymbol{\beta}$ is the $M \times 1$ vector of effect sizes for each variant (to be estimated)
\item $\boldsymbol{\epsilon}$ is the $N \times 1$ vector of error terms for $N$ individuals and $\boldsymbol{\epsilon} \sim N(0, \sigma^2\mathbf{I})$
\end{itemize}

Using \textbf{ordinary least squares (OLS)}, we can derive the estimators for $\boldsymbol{\beta}$ in matrix form:

$$
\hat{\boldsymbol{\beta}}_{\text{OLS}} = (\mathbf{X}^T\mathbf{X})^{-1}\mathbf{X}^T\mathbf{Y}
$$

\end{frame}


\begin{frame}{Technical Details: Marginal Effect}
In the section about OLS, we discuss the single marker linear regression, which estimates the marginal effect of each X.

The marginal effect of a genetic variant is its association with the trait when analyzed in isolation, without accounting for other variants:

$$
\hat{\beta}_{\text{marginal},j} = (\mathbf{X}_j^T\mathbf{X}_j)^{-1}\mathbf{X}_j^T\mathbf{Y}
$$

Where $\mathbf{X}_j$ is the column vector for the $j$-th variant.

\end{frame}

\begin{frame}{Technical Details: Joint Effect}
The joint effect of a genetic variant is its association with the trait when analyzed simultaneously with other variants, i.e., in the multiple marker model:

$$
\hat{\boldsymbol{\beta}}_{\text{joint}} = (\mathbf{X}^T\mathbf{X})^{-1}\mathbf{X}^T\mathbf{Y}
$$

Where $\hat{\beta}_{\text{joint},j}$ (the $j$-th element of $\hat{\boldsymbol{\beta}}_{\text{joint}}$) represents the effect of the $j$-th variant after accounting for all other variants in the model.

\end{frame}

\begin{frame}{Technical Details: Key Differences Between Marginal and Joint Effects}

1. \textbf{Correlation Structure}: 
\begin{itemize}
\item Marginal effects ignore correlations (linkage disequilibrium) between variants
\item Joint effects account for correlations between variants
\end{itemize}

2. \textbf{Interpretation}:
\begin{itemize}
\item Marginal effect: The expected change in the trait associated with a unit change in the variant, not accounting for other variants
\item Joint effect: The expected change in the trait associated with a unit change in the variant, holding all other variants constant
\end{itemize}

3. \textbf{Consistency}:
\begin{itemize}
\item When variants are uncorrelated, marginal and joint effects are identical
\item When variants are correlated, marginal and joint effects will differ
\item Joint effects can be smaller or larger than marginal effects, or even have opposite signs
\end{itemize}
\end{frame}

