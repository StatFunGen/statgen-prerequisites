% Generated slides for Genotype Coding
% Include this file in your main Beamer presentation

\begin{frame}{Section}
\centering
\Huge{Confounder}
\end{frame}


\begin{frame}{Intuitional Description}
A confounder is a variable that \textbf{influences both the exposure and outcome independently}, creating a \textbf{misleading association} between them that doesn't represent a true causal relationship.
\end{frame}

\begin{frame}{Graphical Summary}
\includesvg[width=0.8\textwidth]{./cartoons/confounder.svg}
\end{frame}


\begin{frame}{Key Formula}
The key formula for the concept of a confounder is represented in a causal diagram as:
$$
X ← C → Y
$$
Where:
\begin{itemize}
\item $C$ is the confounder variable
\item $X$ is the exposure/treatment variable 
\item $Y$ is the outcome variable
\item The arrows (←, →) indicate the direction of causal influence
\end{itemize}

This diagram illustrates that a confounder ($C$) has a direct causal effect on both the exposure ($X$) and the outcome ($Y$), creating a "backdoor path" between $X$ and $Y$ that must be blocked to obtain an unbiased estimate of the causal effect.

\end{frame}


\begin{frame}{Technical Details}
Confounding can cause bias in the estimation of the relationship between the genotype and the trait. To avoid this bias, confounders \textbf{must be controlled} in the analysis, typically by including them as covariates in regression models.

In formal causal inference terminology, a confounder creates a situation where:
$$
P(Y|X) \neq P(Y|\text{do}(X))
$$
Where do(X) represents an intervention to set X to a specific value. This inequality shows that the observed association differs from the true causal effect due to the confounding variable.

When adjusting for confounders in statistical models:
1. Stratification: Analyzing the $X$-$Y$ relationship separately within strata of C
2. Regression adjustment: $Y = \beta_0 + \beta_1 X + \beta_2 C + \epsilon$
3. Propensity score methods: Creating balanced groups based on $P(X=1|C)$
4. Instrumental variables: Using a variable $Z$ where $Z→X→Y$ and $Z \perp \!\!\! \perp C$
5. Directed Acyclic Graphs (DAGs): Identifying minimal sufficient adjustment sets

The backdoor criterion in causal inference provides a graphical rule for identifying which variables need to be controlled to eliminate confounding bias when estimating causal effects.
\end{frame}

