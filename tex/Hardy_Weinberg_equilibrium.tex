% Generated slides for Genotype Coding
% Include this file in your main Beamer presentation

\begin{frame}{Section}
\centering
\Huge{Hardy Weinberg Equilibrium}
\end{frame}


\begin{frame}{Intuitional Description}
The Hardy-Weinberg equilibrium (HWE) describes how allele and genotype frequencies remain constant across generations in a population where there is no interference from evolutionary forces.
\end{frame}

\begin{frame}{Graphical Summary}
\includesvg[width=0.8\textwidth]{./cartoons/Hardy_Weinberg_equilibrium.svg}
\end{frame}


\begin{frame}{Key Formula}
For a genetic variants with two alleles (\texttt{A} and \texttt{a}) with frequencies $p$ and $q$ respectively (where $p + q = 1$):

$$p^2 + 2pq + q^2 = 1$$

Where:
\begin{itemize}
\item $p^2$ = frequency of genotype AA
\item $2pq$ = frequency of genotype Aa
\item $q^2$ = frequency of genotype aa
\end{itemize}
\end{frame}


\begin{frame}{Technical Details: Expected Counts Under HWE}
The expected counts under HWE are:

$$
E_{AA} = f_A^2 \cdot N
$$

$$
E_{Aa} = 2f_A f_a \cdot N
$$

$$
E_{aa} = f_a^2 \cdot N
$$

where:  
\begin{itemize}
\item $f_A$: frequency of allele A
\item $f_a = 1 - f_A$: frequency of allele a
\item $N$ = Total number of individuals.
\end{itemize}

\end{frame}

\begin{frame}{Technical Details: Test HWE Using Chi-squared Test}

Then one can use Pearson's chi-squared test to test if HWE holds:

$$
\chi^2 = \sum \frac{(O_i - E_i)^2}{E_i}
$$

where:  
\begin{itemize}
\item $O_i$ = Observed genotype count (AA, Aa, aa).  
\item $E_i$ = Expected genotype count under Hardy-Weinberg Equilibrium.  
\item The summation runs over all genotype categories.  
\end{itemize}

To determine statistical significance, compare $\chi^2$ with a \textbf{chi-square distribution} with \textbf{1 degree of freedom} (df = 1). The p-value is computed as:

$$
p = P(\chi^2 > \text{observed } \chi^2)
$$

If $p < 0.05$, we reject the \textbf{Hardy-Weinberg Equilibrium} assumption.
\end{frame}

\begin{frame}{Technical Details: When HWE Doesn't Hold}
\begin{itemize}
\item \textbf{Non-random mating}: When individuals choose mates based on genotype or phenotype, homozygosity increases beyond HWE expectations.
\item \textbf{Natural selection}: When certain genotypes have survival or reproductive advantages, their frequencies change between generations.
\item \textbf{Migration}: Gene flow introduces new alleles from other populations, altering local allele frequencies.
\item \textbf{Genetic drift}: Random sampling effects in small populations cause unpredictable changes in allele frequencies.
\item \textbf{Mutation}: New alleles emerge through mutation, changing the genetic composition of the population.
\end{itemize}

\end{frame}

\begin{frame}{Technical Details: Common Misconceptions}
\begin{itemize}
\item While entire genomes aren't in HWE, individual loci often are, especially neutral markers.
\item HWE is surprisingly robust to minor violations of its assumptions.
\item Deviations from HWE often signal important biological phenomena rather than errors.
\item HWE describes the mathematical relationship between allele and genotype frequencies, not the absence of genetic variation.
\end{itemize}

\end{frame}

\begin{frame}{Technical Details: Role of HWE in Statistical Genetics}

Hardy-Weinberg Equilibrium serves several critical functions in statistical genetics:

\begin{itemize}
\item \textbf{Quality control baseline}: Significant deviations from HWE often signal genotyping errors or technical artifacts rather than biological phenomena, providing an efficient method to identify problematic markers.
\end{itemize}

\begin{itemize}
\item \textbf{Null hypothesis framework}: HWE establishes the expected genotype distribution under neutral conditions, serving as the statistical null model against which evolutionary forces can be detected.
\end{itemize}

\begin{itemize}
\item \textbf{Allele frequency estimation}: When only partial genotype data is available, HWE principles allow researchers to estimate complete population allele frequencies.
\end{itemize}

\begin{itemize}
\item \textbf{Statistical power improvement}: Filtering out markers that violate HWE improves signal-to-noise ratio in association studies, increasing power to detect true genetic effects.
\end{itemize}

\begin{itemize}
\item \textbf{Population structure inference}: Systematic HWE deviations across multiple loci can reveal cryptic population substructure that might confound genetic analyses.
\end{itemize}
\end{frame}

