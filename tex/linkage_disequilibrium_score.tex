% Generated slides for Genotype Coding
% Include this file in your main Beamer presentation

\begin{frame}{Section}
\centering
\Huge{Linkage Disequilibrium Score}
\end{frame}


\begin{frame}{Intuitional Description}
The LD score of a variant is a measure to describe how connected or tagged this variant is with other variants by summing up the squared correlation (r²) with every other variant under consideration.

\end{frame}

\begin{frame}{Graphical Summary}
\includesvg[width=0.8\textwidth]{./cartoons/linkage_disequilibrium_score.svg}
\end{frame}


\begin{frame}{Key Formula}
The LD score for a SNP is the sum of LD $r^2$ measured with all other SNPs:

$$
l_j = \sum_{k=1, k \neq j}^M r^2(\textbf{X}_j, \textbf{X}_k)
$$
\end{frame}


\begin{frame}{Technical Details}
The \textbf{LD score} is a measure of the extent to which a given variant is in linkage disequilibrium (LD) with other variants across the genome. It is used to summarize the amount of genetic information (in terms of LD) that a variant shares with all other variants in a region of interest. The LD score for a variant $j$ is defined as the sum of the squared correlation coefficients $r^2$ between that variant and all other variants in the genome, typically within a specified genomic window or region. Mathematically, the LD score for variant $j$ is given by:

$$
l_j = \sum_{k=1, k \neq j}^M r^2_{i,k}
$$

<!-- Since $\textbf{X}$ is standardized, we can just calculate the sum of the squared sample correlations like this:

$$\widetilde{l}_{j} = \frac{1}{N^2}\textbf{X}^\top_j\textbf{X}\textbf{X}^\top\textbf{X}_j - 1$$ -->
However, this is not an unbiased estimate. We can correct for the bias like this:

$$
r_\text{adj}^2 =  \hat{r}^2 - \frac{1-\hat{r}^2}{N-2}
$$
$$
l_j = \sum_{k=1, k \neq j}^M  r_\text{adj}^2 
$$

This score reflects how much a variant is correlated with other variants across the genome, providing insight into the local structure of LD around that variant.

\end{frame}

